\documentclass[10pt,a4paper,twoside]{article}
\usepackage[dutch]{babel}
\usepackage{amssymb}
\usepackage{amsmath}
\usepackage{float,flafter}	
\usepackage{hyperref}
\usepackage{inputenc}
\setlength\paperwidth{20.999cm}\setlength\paperheight{29.699cm}\setlength\voffset{-1in}\setlength\hoffset{-1in}\setlength\topmargin{1cm}\setlength\headheight{12pt}\setlength\headsep{0cm}\setlength\footskip{1.131cm}\setlength\textheight{25cm}\setlength\oddsidemargin{2.499cm}\setlength\textwidth{15.999cm}
\begin{document}
\begin{center}
\vspace{.3cm}
{\bf {\huge Application of ORM }}
\vspace{.3cm}
\end{center}
{\bf Name:}  Bhavin Raichura\\
{\bf Roll no:}  19111019 \\
{\bf Subject:}  DBMS\\
{\bf Branch: }  Biomedical Engineering \hspace{\fill}   \\
\hrule

\vspace{.5cm}
\vspace{.4cm}

\renewcommand{\abstractname}{Abstract}
Object-oriented applications often achieve persistence by using relational database systems. In such setup, object- relational mapping is used to link objects to tables. Due to fundamental differences between object-orientation and re- lational algebra, the definition of a mapping is a consider- ably difficult task. Today, there are only informal guidelines that support engineers in choosing the best mapping strategy. However, guidelines do not provide a quantification of actual impact and trade-off between different strategies. Thus, the decision on which mapping strategy should be implemented relies on a large portion of gut feeling. In this paper, we propose a framework and conduct a quan- titative study of the impact of object-relational mapping strate- gies on selected non-functional system characteristics. Our study creates awareness for consequences of using different mapping designs and persistence technologies. This allows developers to make distinctive and informed decisions, based on quantified results rather than gut feeling.

\begin{abstract}

\end{abstract}


\end{document}
